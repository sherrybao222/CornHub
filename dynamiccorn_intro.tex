\section{Overview}
Modeling is a powerful tool in synthetic biology and engineering. Modeling has provided us with an important engineering approach to characterize our pathways and predict their performance, thus helped us with modifying and testing our designing.

Basically, the models built by us can be divided into two parts.On kinetic model parts, we hope to gain insight of the gene expression dynamics of our whole circuit. And also we tried to better characterize our parts, analyze our experimental data,and protein transport and concentration changes throughout the whole process. Several tools including ODEs and interpolation are employed.
	
On the other parts.........
	
\subsection{Kinetic model}
  
\subsubsection{transcription}
	
\begin{equation}
plasmid \stackrel{k_1}{\longrightarrow} mitoCas9 
\end{equation}

\begin{equation}
plasmid\stackrel{k_2}{\longrightarrow}sgRNA
\end{equation}

\subsubsection{bonding}

\begin{equation}
mitoCas9+sgRNA\stackrel{k_3}{\longrightarrow}mitoCas9/sgRNA_F
\end{equation}

\subsubsection{entering mitochondria}
There is a small model for the process of mitoCas9 protein entering mitochondria nested in it.

The reaction equations for the process is under the leadership of MTS presequences, which can bind to the outer membrane of mitochondria.After the presequence binds to the mitochondrial surface, its subsequent translocation can be considered to be analogous to a unireactant enzymatically catalyzed process, except that all the steps take place within the confines of the mitochon­ drial membranes. This binding remains in equilibrium dur­ingthe import and is described by  the partition  coefficient,  $r (liter/ 
m^2)$,  refers to the ratio between the concentration of bound presequence (with re­spec to the surface of the mitochondria) and the concentra­tio nof free presequence. 

\begin{equation}
P_F\rightleftharpoons\ P_B
\end{equation}

\begin{equation}
r=\frac{[P_B]_S}{[P_F]}
\end{equation}

\begin{equation}
%P_B+E \xrightleftharpoons[k_{-1}]{k_1}  E * P_B \stackrel{k_2}{\longrightarrow} P_I+E
\end{equation}

The concentrations of bound presequences or of intrinsic, membranebound proteins that are defined relative to the area of the external mitochondrial surface will be noted by a subscript or superscript S (for example, $[P_Bl_S$).The value $C_M$ is the concentration of mitochondria (g/liter), and Ks (m2/g) is a proportionality factor that relates the surface area of the outer membrane of the mitochondria to the amount of mitochondrial protein. So that

\begin{equation}
[P_B]_S=\frac{[P_B]}{K_S C_M}
\end{equation}
Briggs-Haldane steady-state assumption:
\begin{equation}
K_M^S=\frac{k_1+k_2}{k_1}=\frac{[E]_S [P_B]_S}{[E*P_B]_S}
\end{equation}
as
\begin{equation}
\frac{[E]_S}{[E*P_B]_S}=\frac{[E]}{[E*P_B]}
\end{equation}
then
\begin{equation}
[E]=\frac{K_M^S[E*P_B]}{[P_B]_S}
\end{equation}
as
\begin{equation}
[E_T]=[E]+[E*P_B]
\end{equation}
then
\begin{equation}
[E*P_B]=\frac{[E]_T[P_B]_S}{K_M^S+[P_B]_S}
\end{equation}
also
\begin{equation}
E_T=K_EC_M
\end{equation}
$K_E$(mol of translocator per g of mitochondrial protein) \\
then
\begin{equation}
[E*P_B]=\frac{EC_M[P_B]}{K_M^SC_MK_S+[P_B]}
\end{equation}
\begin{equation}
\frac{d([P_F]+[P_B])}{d t}=-k_2[E*P_B]
\end{equation}
From the binding experiments, the relationship between $[P_F]$ and $[P_B]$ is known:
\begin{equation}
[P_F]+[P_B]=[P_B](\frac{C_M^{50}}{C_M}+1)
\end{equation}
then
\begin{equation}
C_MK_M^SK_S \frac{d[P_B]}{P_B}+d[P_B]=-k_2K_E(C_M)^2(C_M^{50}+C_M)^{-1} d t
\end{equation}
Integration of this expression over the range $[P_B]_0$ to $[P_B]$ yields:
\begin{equation}
C_MK_M^SK_S\ln\frac{[P_B]_0}{[P_B]}+([P_B]_0-[P_B])=\frac{k_2K_E(C_M)^2 t}{(C_M^{50}+C_M)}
\end{equation}
However, no unique convergent fit could be obtained. Then we applied the same kinetic data to a simplified version of the integrated rate equation that was appropriate for subsaturating $P_B$. In that case,  $[E] \approx [E]_T$, and $[E*P_B]$ could be neglected in the conservation-of-mass relationship. Then:
\begin{equation}
K_M^SK_S\ln\frac{[P_B]_0}{[P_B]}=k_2K_EC_Mt(C_M^{50}+C_M)
\end{equation}
It should be noted that Eq.(13) is the limit of Eq.(12) when $C_MK_S K_sln([P_Bl_0/[P_B]) >> ([P_B]_0 - [P_Bl)$. The data from the presequence imports gave a good fit to this equation with $k_2K_E(K_s K_s)^{-1} = 0.19 min^{-1}$.
	\\\\
	The symbol declaration is:\\\\
	$P_F$: presequence free in the external  solution\\
	$P_B$: presequence bound externally to the outer membrane of the mitochondria\\
	$E$ : a translocator that transfers presequence from the outer membrane into the mitochondria\\
	$C_M$: the concentration of mitochondria (g/liter)\\
	$K_s$: a proportionality factor that relates the surface area of the outer membrane of the mitochondria to the amount of mitochondrial protein (m2/g) \\
	$E_T$: the total bulk molar con entration of the translocator\\
	$K_E$: mol of translocator per g of mitochondrial protein
	\\\\ 
	
\subsubsection{incising}

\begin{displaymath}
mitoCas9/sgRNA_I+target\longrightarrow mitoCas9/sgRNA/target\stackrel{k_4} {\longrightarrow} mitoCas9/sgRNA_I+detarget
\end{displaymath}

In this expression, “$[mitoCas9/sgRNA_I]$” is $ [mitoCas9/sgRNA_B]_0-[mitoCas9/sgRNA_B]$ 

\subsection{solutions and implication}

\begin{equation}
\frac{d[plasmid]}{dt}=-k_1[plasmid]-k_2[plasmid]\tag{1}
\end{equation}
\begin{equation}
\frac{d[mitoCas9]}{dt}=k_1[plasmid]-k_3[mitoCas9][sgRNA]\tag{2}
\end{equation}
\begin{equation}
\frac{d[sgRNA]}{dt}=k_2[plasmid]-k_3[mitoCas9][sgRNA]\tag{3}
\end{equation}
\begin{equation}
\frac{d[mitoCas9/sgRNA_F]}{dt}=k_3[mitoCas9][sgRNA]\tag{4}
\end{equation}
\begin{equation}
 r=\frac{[mitoCas9/sgRNA_B]}{[mitoCas9/sgRNA_F]K_S C_M}\tag{5}
\end{equation}
\begin{equation}
K_M^SK_S\ln\frac{[mitoCas9/sgRNA_B]_0}{[mitoCas9/sgRNA_B]}=k_2K_EC_Mt (C_M^{50}+C_M)\tag{6}
\end{equation}
\begin{equation}
 [mitoCas9/sgRNA_I]= [mitoCas9/sgRNA_B]_0-[mitoCas9/sgRNA_B]\tag{7}
\end{equation}
\begin{equation}
\frac{d[target]}{dt}=-\frac{[mitoCas9/sgRNA_I][target]}{k_4+[target]}\tag{8}
\end{equation}

